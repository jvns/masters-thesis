Le calcul quantique topologique est une approche au probl\`eme d'implementation
de circuits quantique d'une fa\c{c}on robuste et precis\'e.  L'id\'ee s'agit
d'exploiter certaines propri\'et\'es de quasiparticules, dites "anyons", pour
obtenir une impl\'ementation du calcul quantique qui est intrinsequement
tolerante aux pannes. La structure math\'ematique qui d\'ecrit ces anyons est celle
des cat\'egories modulaires.  Ces objets peuvent \^etre construites \`a partir de
repr\'esentations de certaines alg\`ebres, appel\'ees groupes quantiques.  Dans ce
m\'emoire, nous donnerons une exposition des cat\'egories modulaires, des groupes
quantiques et du lien qu'ils partagent avec le calcul quantique. Le m\'emoire ne
devrait requ\'erir qu'une connaissance de base en alg\`ebre et en th\'eorie des
categories. L'espoir \'etant de donner un model concret pour les informaticiens
de la th\'eorie de cat\'egories obtenus \`a partir de groupes quantiques.  L'emphase
sera sur le point de vu alg\`ebrique et cat\'egorique plut\^ot que celui physique.
