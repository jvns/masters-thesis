
\section{The Fibonacci anyon}
The simplest universal model of TQC, and so the most likely candidate for
implementing TQC, is the Fibonacci anyon. We will introduce the Fibonacci
anyon, and show how this model comes from the modular tensor category obtained
from $U_q(\sll(2))$ for $q=e^{2\pi i/5}$ in some detail. We will also explain a
fairly simple way to simulate quantum circuits with Fibonacci anyons, where
fusion possibilities \todo{better term?} are the qubits.

In the Fibonacci anyon model there are 2 types, labelled $\mathbf{1}$, $\tau$. $\mathbf{1}$ is the trivial type. The fusion rules are as follows: 


\begin{itemize}
    \item $\tau \odot \tau = I \oplus \tau$
    \item $\tau \odot I = \tau$
    \item $I \odot I = I$
\end{itemize}

As with all TQC theories, this takes place in a MTC which we call $\Fib$. The
MTC $\Fib$ has two isomorphism classes of simple objects labelled by
$\mathbf{1}$, $\tau$. $\mathbf{1}$ is the tensor unit. We will denote the
tensor product by $\odot$. The $\oplus$ in the first fusion rule above should
be taken to mean that if two anyons of type $\tau$ fuse, then the result is an
anyon of type either $\mathbf{1}$ or $\tau$ (or a superposition). If we compute
$\tau \odot (\tau \odot \tau)$, we get
\begin{equation}
    (\tau \odot \tau) \odot \tau = (\tau \oplus I) \odot \tau = (\tau \odot \tau) \oplus (I \odot \tau) = (\tau \oplus I) \oplus \tau = 2\tau \oplus I
\end{equation}

This means that when $\tau,\tau,\tau$ fuse, the result is either $\tau$ in one of
2 possible ways, or $\mathbf{1}$ (in one way)''. What do we mean by ``in one of two
possible ways?'' Well, there are 3 fusion possibilities for 3 anyons, which we
can draw as follows 
\begin{center}
    (draw some fusion diagrams here)
    \todo{make these diagrams}
\end{center}

So in the first and third case the end result is $\tau$, and in the second case
$\mathbf{1}$. So we see that there are two different ways of obtaining a final
result of $\tau$: the first two anyons can fuse with a result of either $\tau$
or $\mathbf{1}$.


\subsection{Where are the qubits?}

When three anyons of type $\tau$ are fused, there are 3 possibilities for the result: 

\begin{center}
    (draw some fusion diagrams here)
    \todo{make these diagrams}
\end{center}

2 of them have the final result $\tau$. Another way of putting this is 

\begin{equation}
    \Hom(\tau, \tau^{\odot 3}) = \mathbb{C}^2
\end{equation}



\section{Universality}
