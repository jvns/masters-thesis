One of the major challenges in implementations of quantum computing is that of
implementing quantum gates accurately and robustly. Topological quantum
computing exploits topological properties of certain quasiparticles called
anyons to obtain a proposed implementation of quantum computing which is
inherently fault-tolerant. The mathematical structure that describes anyons is
that of modular tensor categories. These modular tensor categories can be
constructed from the representations of certain algebraic objects called
quantum groups. We give an explanation of modular tensor categories and quantum
groups as they relate to topological quantum computing. It is intended that
this thesis can be read with some basic knowledge of algebra and category
theory.  The hope is to give a concrete account accessible to computer
scientists of the theory of modular tensor categories obtained from quantum
groups. The emphasis is on the category theoretic and algebraic point of view
rather than on the physical point of view. 
