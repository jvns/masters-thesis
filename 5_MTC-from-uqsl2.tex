% from Bakalov & Kirillov, section 3.3
We will denote the category of representations of $U_z^{res}(\mathfrak{g})$ at
a $\ell^\text{th}$ root of unity $z$ by $\mathcal{C}(\mathfrak{g}, \ell)$. We
know that $\mathcal{C}(\sll(2), \ell)$ is a rigid monoidal category since
$U_z^{res}(\sll(2))$ is a Hopf algebra. $\mathcal{C}(\sll(2), \ell)$ is also
braided, as we saw in \ref{section:braiding}. However, we saw in
\ref{section:RepTheoryofResSL2} that $\mathcal{C}(\sll(2), \ell)$ is not
semisimple: the Weyl modules $W_z^{res}(n)$ are not in general completely
reducible.   We will discuss in this section how to obtain a modular tensor
category from $\mathcal{C}(\sll(2), \ell)$. this category's simple objects will
be our anyons, and each such category is a topological quantum computing
theory.
\todo{Why is it a Hopf algebra? Show this somewhere.}



\section{Quantum trace}

The Hopf algebra structure on $U_q(\mathfrak{g})$, together with the braiding
discussed in the last section gives the category of representations
$\mathcal{C}(\mathfrak{g}, \ell)$ the structure of a ribbon category. There is
therefore a well-defined trace of an endomorphism in the category, which is
different from the usual trace of an vector space endomorphism. 

We will denote the trace of a morphism $f$ by $\tr_q(f)$.

In the case $U_q(\sll(2))$, the quantum trace is given by 

\begin{equation}
    \tr_q(f) = \tr(Kf)
\end{equation}

where $\tr$ is the usual trace of a vector space endomorphism.

For any object $M$ of $\mathcal{C}(\mathfrak{g}, \ell)$, we define the quantum dimension $\dim_q(M)$ to be 

\begin{equation}
    \dim_q(M) = \tr(\id_M)
\end{equation}

For example, we can see from the definition of the Weyl module $W_z^{res}(n)$
that $\dim_q(W_z^{res}(n)) = [n+1]_z$. In particular we see that
$\dim_q(W_z^{res}(\ell - 1)) = 0$, so objects other than the zero object can
have quantum dimension zero. 

\section{Tilting modules} 
% when citing, say something like: originally done by Anderson & P, there's
% something more accessible by Sawin


\begin{defn}
    Suppose $M$ is a $U_q(\sll(2))$-module. A \emph{Weyl filtration} for $M$
    is a sequence of submodules 

    \begin{equation}
        \left\{ 0 \right\} = J_0 \subset \cdots \subset J_n = M
    \end{equation}

    such each $J_k$ is a maximal submodule of $J_{k+1}$ and each quotient $J_{k+1}/J_k$ is a Weyl module. 
\end{defn}

\begin{defn}
    A $U_q(\sll(2))$-module $M$ is called \emph{tilting} if both $M$ and $M^*$ have Weyl filtrations.
\end{defn}

The following proposition lists some properties of tilting modules which will
be important.

\begin{prop}
\begin{enumerate}
    \renewcommand{\labelenumi}{\roman{enumi})}
    \item The dual of any tilting module is tilting.
    \item The direct sum of two tilting modules is tilting
    \item Any direct summand of a tilting module is tilting
    \item The tensor product of two tilting modules is tilting
\end{enumerate}
\end{prop}

We can therefore restrict our attention to the indecomposable tilting modules.
The indecomposable tilting modules for $U_z^{res}$ are indexed by integers $n
\geq 0$. The tilting modules $T_z(n)$ for $0 \leq n < 2\ell - 2$ can be
described explicitly as follows.

$T_z(n)$ has basis $\left\{ t_0, \ldots, t_n \right\} \bigcup \left\{ t'_0,
    \ldots, t'_{r} \right\}$, where $r = 2\ell - 2 - n$, and has the following
    action:

\begin{align*}
    Kt_i &= z^{n-2i} t_i \\
    X^+ t_i &= [n-i+1]_z t_{i-1} \\
    (X^+)^{(\ell)} t_i &= ( (n-i)_1) + 1) t_{i-\ell} \\
    X^-t_i &= [i+1]_z t_{i+1} \\
    (X^-)^{(\ell)} t_i &= ( i_1 + 1) t_{i+\ell} \\
    Kt'_i &= z^{r-2i} t'_i \\
    X^+ t'_i &= [r-i+1]_z t'_{i-1}  + \dbinom{n + i - \ell}{i}_z t_{n+i-\ell} &\text{ for $0 < i \leq r$}\\
    X^+ t'_0 &= [n - \ell + 1]_z t_{n-\ell} \\
    X^-t'_i &= [i+1]_z t'_{i+1} &\text{ for $0 \leq i < r$}\\
    X^- t'_r &= \dbinom{\ell - 1}{n - \ell + 1}_z t_\ell \\
    (X^\pm)^{(\ell)} t'_i  &= 0
\end{align*}

\todo{put a nice picture here}

It is easy to see that the $T_z(n)$ are indecomposable, and that $\left\{ t_0,
\ldots, t_n \right\}$ span a submodule isomorphic to $W_z^{res}(n)$. The
quotient is spanned by $\left\{ t'_0, \ldots, t'_r \right\}$ and is
isomorphic to $W_z^{res}(r)$. It is therefore clear that $T_z(n)$ has a Weyl
filtration. 

Given this definition, it is easy to compute the quantum dimension of $T_z(n)$: 

\begin{equation}
    \dim_q(T_z(n)) = \begin{cases} [n+1]_z &\text{ for $0 \leq n < \ell$} \\
                                   [n+1]_z + [2\ell-n-1]_z = 0 &\text{ for $\ell \leq n < 2\ell - 2$}
                     \end{cases}
\end{equation}

In fact, \cite{andersen1992} shows that 
\begin{prop}
$\dim_q(T_z(n)) \neq 0$ if and only if $n < \ell - 1$
\end{prop}

which gives us $\dim_q(T_z(n))$ for all $n$.

We would like to know how to decompose tensor products of irreducible modules. 

\begin{prop}
    For any tilting modules $T_1, T_2$, 

    \begin{equation}
            T_1 \otimes T_2 \simeq \left(\bigoplus_{n=0}^{\ell - 2} V_z^{res}(n)^{\otimes m_n}\right) \oplus Z
    \end{equation}
    where $m_n \in \mathbb{Z}$, and $\tr_q(f) = 0$ for all $f: Z \to Z$.
\end{prop}
\begin{proof}
%todo
\end{proof}
%This tensor product will not decompose as a direct
%sum of irreducible modules in general as with the irreducible representations
%of $\sll(2)$, but it does decompose as a direct sum of irreducible modules and
%tilting modules $T$ with $\dim_q(T) = 0$.

 With the definition of these
tilting modules in place, we can give the decomposition of the tensor products
of the irreducible $U_z^{res}$-modules $V_z^{res}(n) \otimes V_z^{res}(m)$ for
$0 \leq m,n \leq \ell - 1$. 

\begin{prop}
Suppose $0 \leq m,n \leq \ell - 1$. Then

\begin{equation}
V_z^{res}(n) \otimes V_{z}^{res}(m) \simeq \bigoplus_{i=|n-m| \atop i + n + m \text{ even}}^{n+m} V_z^{res}(i) 
\end{equation}

if $n+m \leq \ell - 2$, and 

\begin{equation}
V_z^{res}(n) \otimes V_{z}^{res}(m) \simeq \bigoplus_{i=|n-m| \atop i + n + m \text{ even}}^{2\ell - 4 - n - m} V_z^{res}(i) 
                                    \oplus \bigoplus_{i = \ell - 1 \atop i + n + m \text{ even}} ^{n + m} T_z(i)
\end{equation}

for $\ell - 1 \leq n + m < 2\ell - 2$
\end{prop}

We will now construct a new tensor product $\bar{\otimes}$ by ``discarding''
those indecomposable tilting modules $T$ for which $\dim_q(T) = 0$.


Let $\tilt$ be the full subcategory of $\mathcal{C}(\sll(2), \ell)$ whose
objects are the tilting modules modules for $U_z^{res}$. Define 2 full
subcategories of $\tilt$: $\overline{\tilt}$ consisting of the
irreducible tilting modules $\{T_z(0), \ldots, T_z(\ell - 2)\}$ and $\tilt'$
consisting of the tilting modules with quantum dimension zero. 
\todo{figure why the $\ell$ isn't showing up.}

Then $\tilt= \tilt' \oplus \overline{\tilt}$. \todo{why?}


\section{Construction of the MTC}
\label{MTC-construction}

We construct the modular tensor category that we will use to do topological
computing by restricting our attention to the subcategory of tilting modules of
$U_q(\mathfrak{g})$, and quotienting by the negligible morphisms. We can do
this because the hom sets $Hom(V,W)$ are vector spaces:

\todo{ change the notation here, I guess}
\begin{defn}
    Define the category $\mathcal{C}^\text{int}$ be the category with objects tilting modules and morphisms 

    \begin{equation*}
        \Hom(V,W) = \Hom_T(V,W) / \text{negligible morphisms}
    \end{equation*}

\end{defn}

The category $\mathcal{C}^\text{int}$ has the following properties:
\begin{enumerate}
    \item An object $T$ is negligible if and only if it is isomorphic to 0 in $\mathcal{C}^\text{int}$
    \item  $\mathcal{C}^\text{int}$ is a ribbon category
    \item Any object $T$ in $\mathcal{C}^\text{int}$ is isomorphic to a direct sum of Weyl modules.
    \item $\mathcal{C}^\text{int}$ is a semisimple abelian category. 
    \item $\dim_{\mathcal{C}^\text{int}} T > 0$ for every $T \not\simeq 0$
\end{enumerate}

$\mathcal{C}^\text{int}$ is in fact a modular tensor category. A proof of this
can be found in \cite{Kirillov} 
 
\section{Fusion rules}

The fusion rules for $U_q(\sll(2))$ are given by 

\todo{ notation is inconsistent here}
\begin{equation}
    V_m \otimes V_n \simeq \sum_i N_{mn}^i V_i
\end{equation}

where 

\begin{equation}
    N_{mn}^i = \begin{cases} 1 \text{ for } |m-n| \leq i \leq m+n, i \leq 2k - (m+n), i + m + n \in 2 \mathbf{Z} \\
                             0 \text{ else } 
               \end{cases}
\end{equation}

% Bakalov & Kirillov


