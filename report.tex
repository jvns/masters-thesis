\documentclass[]{article}
\usepackage{amsmath}
\usepackage{amsthm}
\usepackage{amsfonts}
\usepackage{xypic}

\newtheorem{theorem}{Theorem}[subsection]
\newtheorem{remark}[theorem]{Remark}
\newtheorem{defn}[theorem]{Definition}
\newtheorem{lemma}[theorem]{Lemma}
\newtheorem{example}[theorem]{Example}

\newcommand{\isomto}{\overset{\sim}{\rightarrow}}
\newcommand{\tr}{\operatorname{tr}}
\newcommand{\id}{\operatorname{id}}
\newcommand{\Hom}{\operatorname{Hom}}
\newcommand{\Rep}{\operatorname{Rep}}
\newcommand{\coker}{\operatorname{coker}}
\newcommand{\End}{\operatorname{End}}
\newcommand{\Ob}{\operatorname{Ob}}

\newcommand{\sll}{\mathfrak{sl}}

\numberwithin{equation}{subsection}

\begin{document}
\tableofcontents
\section{Algebraic Background}

The setting for topological quantum computing is a quotient of the category of
representations of a deformation of a semisimple Lie algebra. In this chapter
we define and discuss some relevant results about the required algebraic
structures. Everything in this section takes place over a field of
characteristic 0.

\subsection{Quantum algebra}

    %todo: write this section
\begin{equation}
    [n]_q = \frac{q^n - q^{-n}}{q - q^{-1}} = q^{n-1} + q^{n-3} + \cdots + q^{-n+3} + q^{-n+1}
\end{equation}

\begin{remark}
$q^{2d}=1$ if and only if $[d]_q = 0$, so $[n]_q \neq 0$ for every nonzero integer when $q$ is not a root of unity. 
\end{remark}


\subsection{Representations}

We will require some facts about representations of algebras. 

\subsection{Modules}

    %todo: write this section
In  this section we will be working over a ring $R$ with a unit $1$. 
\begin{defn}
    A $R$-\emph{module} over a ring $R$ is an abelian group $N$ with a binary
    operation $R \times N \to N$ written $r\cdot n$ or $rn$ such that for $r,s \in R$ and $m,n \in N$:

    \begin{enumerate}
        \item r(n + m) = rn + rm
        \item (rs)n = r(sn)
        \item 1n = n
        \item (r+s)n = rn + s
    \end{enumerate}
\end{defn}


\subsection{Lie Algebras}

The examples of quantum groups that we are interested in are obtained as
deformations of semisimple Lie algebras. We will give some basic definitions of
Lie algebras, ideals, simplicity, and semisimplicity, and discuss briefly the
classification of semisimple Lie algebras. Much more detail can be found in
\cite{Humphreys1973}

Throughout this section $\mathfrak{g}$ will denote a Lie algebra.

A \emph{Lie algebra} is a vector space $\mathfrak{g}$ with a bilinear operation $\left[ \cdot, \cdot \right]$  $\mathfrak{g} \times \mathfrak{g} \to \mathfrak{g}$ such that 

\begin{itemize}
    \item $\left[ x,y \right] = -\left[ y,x \right]$
    \item $\left[ x, \left[ y,z \right] \right] + \left[ y, \left[ z,x \right] \right] + \left[ z, \left[ x,y \right] \right] = 0$ (the Jacobi identity)
\end{itemize}

A basic example of a Lie algebra is $\sll(2)$: the algebra of $2 \times 2$ matrices with trace zero. $\sll(2)$ is generated by the matrices 
\begin{equation}
    X = \begin{pmatrix} 0 & 1 \\ 0 & 0 \end{pmatrix},
    Y = \begin{pmatrix} 0 & 0 \\ 1 & 0 \end{pmatrix}, 
    H = \begin{pmatrix} 1 & 0 \\ 0 &-1 \end{pmatrix}
\end{equation}

The subject of general Lie algebras is fascinating, but we will only be
interested here in semisimple Lie algebras, which are a well-characterized
subclass of Lie algebras. In general any finite dimensional Lie algebra over a
field of characteristic zero can be realized as a vector space of matrices such
that 

\[ \left[ X,Y \right] = XY - YX\] 
for any $X,Y$ (Ado's theorem), so the bracket $[X,Y]$ can safely be thought of as the commutator $XY - YX$. 

\begin{defn}
    A subspace $\mathfrak{h} \subset \mathfrak{g}$ of a Lie algebra is a \emph{Lie subalgebra} if $\mathfrak{h}$ is closed under the Lie bracket. 

    If a subalgebra $I$ satisfies further that $[g,I] \subset I$, then we call $I$ a \emph{ideal}.
\end{defn}

\begin{defn}
    A Lie algebra is \emph{simple} if it has no proper ideals and is not commutative. 
\end{defn}

$\sll(2)$ is the lowest-dimensional simple Lie algebra. In fact, it is the
basic example of a simple Lie algebra: the study of its representation theory
is a key factor in the study of the representation theory of all simple Lie
algebras.
%todo: write this better?

\begin{defn}
    A Lie algebra is \emph{semisimple} if it is the direct sum of simple Lie algebras.
\end{defn}

\subsection{Representations of $\sll(2)$}

A \emph{representation} of a Lie algebra $\mathfrak{g}$ is a vector space $V$ together with an action of $\mathfrak{g}$ on $V$ such that 

\begin{align*}
    \left[ x,y \right] v &= x(yv) - y(xv) \\
    (x+y)v &= xv + yv \\
    (ax)v &= a(xv)
\end{align*}

for all $x,y \in \mathfrak{g}, v \in V, a \in \mathbb{C}$

The representations of $\sll(2)$ can be classified as follows: 

For each integer $n \geq 0$, there is a unique (up to isomorphism) representation of $\sll(2)$ of dimension $n+1$. This representation has basis $v_0, \ldots, v_{n}$ such that
% from Kassel, p. 101
\begin{align*}
    &H v_i = (n - 2i) v_i& \\
    &Y v_i = \begin{cases} 
                (i+1)v_{i+1}& \text{ for $i < m$} \\
                0& \text{ for $i = m$} \\
            \end{cases} \\
    &X v_i = \begin{cases} 
                (n-i+1)v_{i-1}& \text{ for $i > 0$} \\
                0& \text{ for $i = 0$} \\
            \end{cases}
\end{align*}



\subsection{Hopf algebras}
\subsubsection{Algebras and Coalgebras}
For a field $k$, a \emph{$k$-algebra} $A$ is a $k$-vector space with an
associative bilinear mapping $A \times A \to A$ which has an identity element
$1 \in A$ such that $1\cdot x = x\cdot 1 = x$ for any $x \in A$.

Put in categorical terms, a $k$-algebra is given by a triple $(A, \mu, \eta)$,
where $A$ is a vector space, and $\mu: A \otimes A \to A$ and $\eta: k \to A$
are linear maps satisfying the axioms:

Associativity:
The diagram
\begin{equation}
\xymatrix{
A \otimes A \otimes A \ar[d]^-{\id \otimes \mu} \ar[r]^-{\mu \otimes \id} & A \otimes A \ar[d]^-{\mu}\\
 A \otimes A \ar[r]^-\mu & A 
}
\end{equation}
commutes.

Unit: 

The diagram

\begin{equation}
    \xymatrix{
    k \otimes A \ar[r]^-{\eta \otimes \id} \ar[rd]_\simeq & A \otimes A \ar[d]^-\mu & A \otimes k \ar[l]_{\id \otimes \eta} \ar[ld]^-{\simeq} \\
    & A &
    }
\end{equation}
commutes.

An algebra is called \emph{commutative} if $x \cdot y = y \cdot x$ for any $x,y\in A$. In other terms, it needs to satisfy the commutativity axiom:

The triangle 

\begin{equation}
    \xymatrix{
    A \otimes A \ar[rr]^-{\tau_{A,A}} \ar[rd]_\mu & & A \otimes A \ar[ld]^-\mu \\
    & A &
    }
\end{equation}

Given two algebras $(A_1, \mu_1, \eta_1)$ and $(A_2, \mu_2, \eta_2)$, a linear
map $f: A_1 \to A_2$ is called a \emph{morphism of algebras} or a
\emph{algebra homomorphism} if $f(\mu_1(a,b)) = \mu_2(f(a), f(b))$ for any
$a,b\in A_1$ and $f(\eta_1(1)) = \eta_2(1)$. In other words, the following two
diagrams need to commute:

\begin{equation}
    \xymatrix{
    1 \ar[r]^-{\eta_1} \ar[rd]_{\eta_2} & A_1 \ar[d]^-f \\
    & A_2
    }
\end{equation}
\begin{equation}
    \xymatrix{
    A_1 \ar[r]^-f & A_2 \\
    A_1 \otimes A_1 \ar[u]^-{\mu_1} \ar[r]_-{f\otimes f} & A_2 \otimes A_2 \ar[u]^-{\mu_2}
    }
\end{equation}


We can obtain the definition of a coalgebra by reversing all the arrows as follows:

\begin{defn}
    A \emph{coalgebra} is a triple $(C, \Delta, \varepsilon)$ where $C$ is a
    vector space, and $\Delta: C \to C \otimes C$, $\varepsilon: C \to k$ are
    linear maps satisfying the axioms:

Associativity:
The diagram
\begin{equation}
\xymatrix{
C \otimes C \otimes C   & C \otimes C \ar[l]_-{\Delta \otimes \id}\\
 C \otimes C\ar[u]^-{\id \otimes \Delta}  & C \ar[l]_-\Delta \ar[u]_-{\Delta}
}
\end{equation}
commutes.

Unit: 

The diagram

\begin{equation}
    \xymatrix{
    k \otimes C  & C \otimes C \ar[l]_-{\varepsilon \otimes \id} \ar[r]^-{\id \otimes \varepsilon} & C \otimes k   \\
    & C \ar[u]_-\Delta \ar[lu]^-\simeq \ar[ru]_-{\simeq}&
    }
\end{equation}
commutes.
\end{defn}
A coalgebra is called \emph{cocommutative} if the triangle 

\begin{equation}
    \xymatrix{
    A \otimes A   & & A \ar[ll]^-{\tau_{A,A}} \otimes A  \\
    & \ar[lu]^-\Delta A \ar[ru]_-\Delta&
    }
\end{equation}

commutes.

A linear map $f: C_1 \to C_2$ between two coalgebras $(C_1, \Delta_1, \varepsilon_1)$, $(C_2, \Delta_2, \varepsilon_2)$ is a coalgebra homomorphism if the following two diagrams commute: 


\begin{equation}
    \xymatrix{
    1 \ar[r]^-{\eta_1} \ar[rd]_{\eta_2} & A_1 \ar[d]^-f \\
    & A_2
    }
\end{equation}
\begin{equation}
    \xymatrix{
    A_1 \ar[r]^-f & A_2 \\
    A_1 \otimes A_1 \ar[u]^-{\mu_1} \ar[r]_-{f\otimes f} & A_2 \otimes A_2 \ar[u]^-{\mu_2}
    }
\end{equation}



\subsubsection{Bialgebras}

Suppose $H$ is a vector space which has both an algebra structure $(H, \mu,
\eta)$ and a coalgebra structure $(H, \Delta, \varepsilon)$. We call this a
\emph{bialgebra} if the two structures are compatible in the following sense:

\begin{defn}
    A vector space with an algebra and coalgebra structure is called a
    \emph{bialgebra} if one of the following two equivalent conditions holds:

    \begin{enumerate}
        \item The maps $\mu$ and $\eta$ are morphisms of coalgebras
        \item The maps $\Delta$ and $\varepsilon$ are morphisms of algebras
    \end{enumerate}
\end{defn}

A morphism of bialgebras is a map which is both a morphism of algebras an a
morphism of coalgebras.
\subsubsection{Hopf Algebras}

% todo: introduce Sweedler notation??

\begin{defn}
    Let $(H, \mu, \eta, \Delta, \varepsilon)$ be a bialgebra. An endomorphism
    $S$ of $H$ is called a \emph{antipode} for the bialgebra if the following
    two diagrams commute:

    \begin{equation}
        \xymatrix{
        H \ar[r]^-{\Delta} \ar[rrd]_-{\varepsilon}& H \otimes H \ar[rr]^-{S \otimes \id} & & H \otimes H \ar[r]^-{\mu} & H \\
        & & k \ar[rru]_-{\eta}& &
        }
    \end{equation}

    \begin{equation}
        \xymatrix{
        H \ar[r]^-{\Delta} \ar[rrd]_-{\varepsilon}& H \otimes H \ar[rr]^-{\id \otimes S} & & H \otimes H \ar[r]^-{\mu} & H \\
        & & k \ar[rru]_-{\eta}& &
        }
    \end{equation}
\end{defn}

\begin{defn}
    A \emph{Hopf algebra} is a bialgebra with an antipode. A morphism of Hopf
    algebras is a morphism between the bialgebras which commutes with the
    antipodes. 
\end{defn}

\begin{example}
    \label{groupalgebra}
    One important example of a Hopf algebra is the Hopf algebra obtained from
    any finite group. 

    Let $G$ be a finite group. The Hopf algebra $k[G]$ is the algebra with
    basis $\left\{ v_g: g \in G \right\}$, multiplication $v_g v_h = v_{gh}$,
    and unit the group identity. 

    We can define a coalgebra structure on $k[G]$ by $\Delta(v_g) = v_g \otimes v_g$
    and $\varepsilon(v_g) = 1$ for any $g \in G$.

    The antipode $S$ is given by $S(v_g) = v_{g^{-1}}$ for any $g \in G$.

    Note that the group algebra is not commutative if the underlying group is
    not commutative, but it is always cocommutative. The examples of Hopf
    algebras we will be interested in later are neither commutative nor
    cocommutative. 
\end{example}

In general the antipode map can be thought of as an analog to the inverse map
in a group. If a Hopf algebra has an antipode, it is unique, and $S^2 = \id$.

\begin{example}
    \label{UnivEnvAlg}
    Given a Lie algebra $\mathfrak{g}$, we can define an associative algebra
    called the \emph{universal enveloping algebra} of $\mathfrak{g}$. The
    universal enveloping algebra of any Lie algebra is a Hopf algebra.

    The \emph{tensor algebra} $T(V)$ of any $k$-vector space $V$ is an associative
    algebra defined by

    \begin{equation}
        T(V) = k \oplus \bigoplus_{n=1}^\infty \underbrace{(V \otimes \cdots \otimes V)}_{\text{$n$ times}}
    \end{equation}

    with multiplication $v \cdot w = v \otimes w$.

    %todo: this doesn't feel well explained :(

    If we have a Lie algebra $\mathfrak{g}$, then we can define an ideal
    $I(\mathfrak{g})$ of the tensor algebra $T(\mathfrak{g})$ generated by all
    elements of the form $(xy - yx) - \left[ x,y \right]$ for $x,y \in
    \mathfrak{g}$.

    We define the universal enveloping algebra to be 

    \begin{equation}
        U(\mathfrak{g}) = T(\mathfrak{g}) / I(\mathfrak{g})
    \end{equation}

    We can put a Hopf algebra structure on the enveloping algebra
    $U(\mathfrak{g})$ for any Lie algebra $\mathfrak{g}$, with $\Delta,
    \varepsilon, S$ defined by: 
    \begin{align}
        \Delta(x) &= x \otimes 1 + 1 \otimes x \\
        \varepsilon(x) &= 0 \\
        S(x) &= -x  
    \end{align}

    for $x \in \mathfrak{g}$.

\end{example}
\subsection{Representations}

\section{Categorical Background}
Throughout we will again be working over a field $k$ of characterisic 0.

\subsection{Monoidal Categories}
\begin{defn}
    A \emph{monoidal category} is a category $\mathcal{C}$ with 
    \begin{enumerate}
    \renewcommand{\labelenumi}{\roman{enumi})}
        \item a bifunctor $\otimes: \mathcal{C} \times \mathcal{C} \to
            \mathcal{C}$
        \item a unit object $\mathbf{1}$ and natural transformations
            \begin{equation}
                \lambda_V : \mathbf{1} \otimes V \isomto V \\
                \rho_V : V \otimes 1 \isomto V
            \end{equation}
        \item a natural transformation 

            \begin{equation}
                \alpha_{UVW} (U \otimes V) \otimes W \isomto U \otimes (V \otimes W)
            \end{equation}

            which satisfy the associativity property 

        \item if $X_1, X_2$ are two objects obtained from $V_1 \otimes V_2 \otimes \cdots V_n$ by inserting 1s and brackets, then all isomorphisms $\varphi: X_1 \isomto X_2$ composed of $\alpha$'s, $\lambda$'s, and $\rho$'s are equal. 
        \item $\mathbf{1}$ is a simple object and $\End_\mathcal{C} \mathbf{1} = k$
    \end{enumerate}

\end{defn}

\begin{example}
    \begin{enumerate}
    \renewcommand{\labelenumi}{\roman{enumi})}
        \item The category of vector spaces over a field $\operatorname{Vec}(k)$
        \item the category of finite dimensional representations of a group, algebra, or Lie algebra
    \end{enumerate}
\end{example}
    
\subsection{Braided monoidal categories}

Let $\mathcal{C}$ be a monoidal category with a natural transformation 

\begin{equation}
    \sigma_{VW} : V \otimes W \to W \otimes V
\end{equation}

\subsection{Rigid monoidal categories}
% todo: write down the vector space example?
A rigid monoidal category is a monoidal category where there is a notion of a dual. 

\begin{defn}
    Let $\mathcal{C}$ be a monoidal category, $V$ an object in $\mathcal{C}$. A \emph{right dual} to $V$ is an object $V^*$ with two morphisms

    \begin{align}
        e_V: V^* \otimes V \to \mathbf{1}  \\
        i_V: \mathbf{1} \to V^* \otimes V
    \end{align}
\end{defn}

such that the composition

\begin{equation}
    V \stackrel{i_v \otimes \id_V}{\xrightarrow{\hspace*{1cm}}} V \otimes V^*
    \otimes V  \stackrel{\id_V \otimes e_V}{\xrightarrow{\hspace*{1cm}}} V
\end{equation}

is equal to $\id_V$, and similarly the composition

\begin{equation}
    V^* \stackrel{id_{V^*} \otimes \id_V}{\xrightarrow{\hspace*{1cm}}} V^*
    \otimes V \otimes V^*  \stackrel{e_V \otimes
    \id_{V^*}}{\xrightarrow{\hspace*{1cm}}} V^*
\end{equation}

is equal to $\id_{V^*}$


\subsection{The Category of Representations of a Hopf Algebra}

Suppose $(H, \mu, \eta, \Delta, \varepsilon)$ is a Hopf algebra with antipode $S$. 

Let $\Rep_f H$ be the category of finite dimensional representations of $H$ as a $k$-algebra.

If $A$ is an algebra and $U, V$ are $A$-modules, then $U \otimes V$ is a vector
space, but there is no natural way to impose a $A$-module structure on $U
\otimes V$. 

The comultiplication $\Delta$ on $H$ allows us to impose a $H$-module structure
on the tensor product $U \otimes V$ of two $H$-modules $U,V$ as follows.

Suppose $\Delta(h) = \sum _{i} h^{(1)}_i \otimes h^{(2)}_i$. Then we define

\begin{equation}
    h (u \otimes v) = \sum_{i} h^{(1)}_i u \otimes h^{(2)}_i v
\end{equation}

We define the tensor unit using the counit $\varepsilon$: $\mathbf{1}$ is the vector space $k$, with 

\begin{equation}
    h(1) = \varepsilon(h) 1
\end{equation}

 for any $h\in H$.

So we have that $\Rep_f(H)$ is a monoidal category, with this tensor product.
Only the counit and the comultiplication are required for this definition, so
in fact the category of representations of any bialgebra is a monoidal
category.

We can use the antipode $S$ to define duals as follows:

For any module $U$, let the dual $U^*$ be the dual vector space of linear functionals on $U$, with action
\begin{equation}
    (h\cdot \varphi)(u)  = \varphi(S(h) u)
\end{equation}

It follows that $\Rep_f(H)$ is a rigid monoidal category. This also serves as
motivation for the definition of a Hopf algebra: it is an algebra with
additional structures such that its category of representations is monoidal and
rigid. 

% maybe say that the category of reps of the group has this structure, and it
% follows that the category of reps of the group algebra k[G] has this
% structure, and the Hopf algebra stuff is a way to generalize that


\subsection{Ribbon Categories}

In what follows it will become important to have a notion of the trace of a
morphism. The definitions that follow will allow us to define a trace in a
certain class of rigid monoidal categories called ribbon categories.

\begin{defn}

    A \emph{ribbon category} is a rigid braided tensor category with a natural isomorphism
    \begin{equation}
        \delta_V: V \to V^{**}
    \end{equation}

such that 
\begin{enumerate}
    \renewcommand{\labelenumi}{\roman{enumi})}

    \item $\delta_{V \otimes W} = \delta_V \otimes \delta_W$
    \item $\delta_1 = \id$
    \item $\delta_{V^*} = (\delta_V^*)^{-1}$
\end{enumerate}

\end{defn}

If $V$ is an object in a ribbon category $\mathcal{C}$ and $f$ an endomorphism of $V$, we can define the trace of $f$ by the composition

\begin{equation}
    \xymatrix{
    \mathbf{1} \ar[r]^-{i_V} & V \otimes V^* \ar[r]^-{f \otimes \id} & V^* \otimes V \ar[r]^-{\delta_V \otimes \id} & V^{**} \otimes V^* \ar[r]^-{e_{V^*}} & 
    \mathbf{1}
    }
\end{equation}

We define the dimension of an object $V$ to be $\dim V = \tr \id_V$.

\subsection{Semisimple Categories}
\begin{defn}
    A category $\mathcal{C}$ is called \emph{abelian} if it satisfies the conditions:

    \begin{enumerate}
    \renewcommand{\labelenumi}{\roman{enumi})}
        \item All the hom sets $\Hom(A,B)$ are $k$-vector spaces, and the compositions
            
            \begin{equation}
                (\varphi, \psi) \mapsto \varphi \circ \psi
            \end{equation}

            are $k$-bilinear.
        \item There is a zero object $\mathbf{0} \in \Ob \mathcal{C}$ such that $\Hom(0,V) = \Hom(V,0) = 0$ for every object $V$
        \item Finite direct sums exist in $\mathcal{C}$
        \item Every morphism $\varphi$ has a kernel $\ker \varphi$ and a
            cokernel $\coker \varphi$. Every morphism is a composition of an
            epimorphism followed by a monomorphism. If $\ker \varphi = 0$, then
            $\varphi = \ker(\coker \varphi)$. If $\coker \varphi = 0$, then
            $\varphi = \coker(\ker \varphi)$.
    \end{enumerate}

    Examples of abelian categories include the category of $k$-vector spaces,
    the category of finite dimensional $k$-vector spaces, and the category of
    representations of a group $G$ over $k$.

\end{defn}

\begin{defn}
    An object $U$ in an abelian category is called $\emph{simple}$ if any
    injection $V \hookrightarrow U$ is either $0$ or an isomorphism.
\end{defn}

\begin{defn}
    An abelian category $\mathcal{C}$ is \emph{semisimple} if any object $V$ is isomorphic to a direct sum of simple objects

    \begin{equation}
        V \simeq \bigoplus{i} N_i V_i
    \end{equation}

    where the $V_i$ are simple objects, $N_i \in \mathbf{N}$

\end{defn}

    Suppose that $\mathcal{C}$ is a semisimple ribbon category. Let $I$ be the
    set of equivalence classes of nonzero simple objects in $\mathcal{C}$ and
    choose a representive $V_i$ for each equivalence class  $i \in I$.
    
    We can define the \emph{fusion coefficients} $N_{ij}^k \in \mathbf{N}$.

    \begin{equation}
        V_i \otimes V_j \simeq \bigoplus_k N_{ij}^k V_k
    \end{equation}

    We call each equation of this type a \emph{fusion rule}. 

    \subsection{Modular Tensor Categories}

The \emph{dimension} of a category $\mathcal{C}$ is defined by 

\begin{equation}
    \operatorname{dim}(\mathcal{C}) = \sum_{ x \in \Ob \mathcal{C}} (\operatorname{dim} x)^2
\end{equation}


\begin{defn}
    The \emph{symmetric center} of a braided category $\mathcal{C}$ $\mathbf{Z}_2(\mathcal{C})$ is a full subcategory with 

    \begin{equation}
        \Ob \mathbf{Z}_2(\mathcal{C}) = \left\{ x \in \mathcal{C} : \sigma_{XY} \circ \sigma_{YX} = \id\ \forall Y \in \mathcal{C} \right\}
    \end{equation}
\end{defn}

\begin{defn}
    A modular tensor category is a semisimple ribbon category such that
    $\mathbf{Z}_2(\mathcal{C})$ is trivial (every object is isomorphic to
    $\mathbf{1}^{\oplus n}$). Equivalently, every simple object in the center
    is isomorphic to $\mathbf{1}$.  
\end{defn}

Why are these categories called modular? 
%todo: finish this 
Define 

\begin{equation}
    s_{X,Y} = \tr_{X \otimes Y}(\sigma_{Y,X} \circ \sigma_{X,Y})
\end{equation}

for every simple object $X,Y$.

\begin{theorem}
    % todo: find a citation for this
    $\mathbf{Z}_2(\mathcal{C})$ is trivial if and only if the matrix $s$ is invertible. 
\end{theorem}

\begin{example}
    An elementary class of examples of modular tensor categories can be obtained via the quantum double construction as follows. 

    % todo: write something about this being a simpler example of something
    % more general that Drinfel'd did

    Recall the Hopf algebra $k[G]$ of any finite group $G$ from \ref{groupalgebra} with basis $\left\{ v_g \right\}_{g \in G}$ and

    % todo: fix aligning here
    \begin{align}
        &\text{multiplication }   & v_g v_h = v_{gh} \\
        &\text{unit }             & e \\
        &\text{comultiplication } & \Delta(v_g) = v_g \otimes v_g \\
        &\text{counit }           & \varepsilon(v_g) = 1 \\
        &\text{antipode }         & S(v_g) = v_{g^-1}
    \end{align}

    The dual Hopf algebra to $k[G]$ is the function algebra $F(G)$ with basis $\left\{ \delta_g : g \in G \right\}$ of functions

    \begin{equation}
        \delta_g(x) = \delta_{g,x} = \begin{cases} 1 &\text{for $g = x$} \\ 0 &\text{for $g \neq x$} \end{cases}
    \end{equation}

    \begin{align}
        &\text{multiplication} &\delta_g \delta_h = \delta_{g,h} \delta_g\\
        &\text{unit}           &\sum_{g \in G} \delta_g \\
        &\text{comultiplication} &\Delta(\delta_g) = \sum_{g_1 g_2 = g} \delta_{g_1} \otimes \delta_{g_2}\\
        &\text{counit}           &\varepsilon(\delta_g) = \delta_{e,g} \\
        &\text{antipode}         &S(\delta_g) = \delta_{g^-1}
    \end{align}


    The quantum double $D(G)$ of $k(G)$ can be described as follows. $D(G)$ is the Hopf algebra with vector space $F(G) \otimes_k k[G]$ and 

    \begin{align}
        &\text{multiplication} &(\delta_g\otimes v_x) (\delta_h \otimes v_y) = \delta_{gx,xh} (\delta_g \otimes v_{xy})\\
        &\text{unit}           &\sum_{g \in G} \delta_g  \otimes v_e\\
        &\text{comultiplication} &\Delta(\delta_g \otimes v_x) = \sum_{g_1 g_2 = g} (\delta_{g_1} \otimes v_x) \otimes (\delta_{g_2} \otimes v_x) \\
        &\text{counit}           &\varepsilon(\delta_g \otimes v_x) = \delta_{e,g} \\
        &\text{antipode}         &S(\delta_g \otimes v_x) = \delta_{x^{-1}g^-1 x} \otimes v_{x^{-1}}
    \end{align}


\end{example}


\section{The Representation Theory of $U_q(\sll(2))$}
\subsection{The quantized universal enveloping algebra $U_q(\sll(2))$}

$U_q(\sll(2))$ is a deformation of the universal enveloping algebra
$U(\sll(2))$ defined in \ref{UnivEnvAlg}, which in some sense approaches
$U(\sll(2))$ as $q \to 1$. In this section we will define $U_q(\sll(2))$ and
make explicit its relationship to $U(\sll(2))$.

% Kassel, p. 121
%% Jantzen, p. 9
Let $q$ be a complex number such that $q \neq 0$, $q^2 \neq 1$.  Then $U_q(\sll(2))$ is the associative algebra with generators $E,F,K, K^{-1}$ and relations 

\begin{align}
    KK^{-1} &= 1 = K^{-1}K \\
    KEK^{-1} &= q^2 E \\
    KFK^{-1} &= q^{-2} F \\
    [E,F] &= \frac{K - K^{-1}}{q - q^{-1}}
\end{align}


%% this presentation taken verbatim from Kassel, p. 125

This is not defined for $q = 1$. We can use an alternate presentation which is
defined for $q = 1$. $U_q(\sll(2))$ is isomorphic to the algebra
$U'_q(\sll(2))$ with generators $E,F,K,K^{-1},L$ and relations:

\begin{align}
    KK^{-1} =\ &1 = K^{-1}K \\
    KEK^{-1} &= q^2 E \\
    KFK^{-1} &= q^{-2} F \\
    [E,F] &= L \\
    (q - q^{-1})L &= K-K^{-1} 
\end{align}

For $q=1$, we can define a homomorphism $\varphi: U'_1(\sll(2)) \to
U(\sll(2))$ by sending $E$ to $X$, $F$ to $Y$, $K$ to 1, and $L$ to $H$. 
\begin{lemma}
This homomorphism is surjective and has kernel generated by $(K-1)$, so
$U'_1(\sll(2)) \stackrel{\sim}{=} U(\sll(2))$
\end{lemma}

%todo: should I explain why the representation theory looks this way? 
% from Kassel, p. 123
There is a unique algebra automorphism of $U_q(\sll(2))$ such that 

\begin{equation}
    \omega(E) = F, \omega(F) = E, \omega(K) = K^{-1}
\end{equation}

This is called the \emph{Cartan automorphism}.

\subsection{A Hopf Algebra Structure on $U_q(\sll(2))$}
%todo: define such a structure on \mathfrak{g}, really
We can define a Hopf algebra structure on $U_q(\sll(2))$ as follows:

% from Kassel, p. 140
\begin{align}
    \Delta(E) &= 1 \otimes E + E \otimes K &  \Delta(F) &= K^{-1} \otimes F + F \otimes 1 \\
    \Delta(K) &= K \otimes K &  \Delta(K^{-1}) &= K^{-1} \otimes K^{-1}\\ 
    \varepsilon(E) &= \varepsilon(F) = 0 &  \varepsilon(K) &= \varepsilon(K^{-1}) = 1
\end{align}

and antipode defined by 

\begin{align}
    S(E) &= -EK^{-1} & S(F)      &= -KF \\
    S(K) &= K^{-1}   & S(K^{-1}) &= K 
\end{align}
It is straightforward to check that this defines a Hopf algebra structure. 

\subsection{Representations of $U_q(\sll(2))$, $q$ not a root of unity}

When $q$ is not a root of unity, the representation theory of $U_q(\sll(2))$ bears a striking resemblance to that of $\sll(2)$. The category of representations of $U_q(\sll(2))$ is semisimple, and  the irreducible representations are classified as follows:

For each integer $n \geq 0$, there are two unique (up to isomorphism)
representations of $U_q(\sll(2))$ of dimension $n+1$. 
The representations are $V(n,+)$ with basis $\left\{ v^+_0, \ldots, v^+_n \right\}$ , and $V(n,-)$ with basis $\left\{ v^-_0, \ldots, v^-_n \right\}$ with action defined by 

% from Kassel, p. 101
\begin{align*}
    &K v^+_i = q^{n-2i} v^+_i  &
    &K v^-_i = -q^{n-2i} v^-_i \\
    &F v^+_i = \begin{cases} [i+1]_qv_{i+1}& \text{ if $i < n$} \\ 0& \text{ if $i = n$} \end{cases} &
    &F v^-_i = \begin{cases} [i+1]_qv'_{i+1}& \text{ if $i < n$} \\ 0& \text{ if $i = n$} \end{cases} \\
    &E v^+_i = \begin{cases} 
                     [n-i+1]_qv_{i-1}& \text{ if $i > 0$} \\ 
                    0& \text{ if $i = 0$} 
             \end{cases} &
    &E v^-_i = \begin{cases} 
                    -[n-i+1]_q v'_{i-1}& \text{ if $i > 0$} \\ 
                    0& \text{ if $i = 0$} 
             \end{cases}
\end{align*}

The proof of this can be done in a way analogous to the classification of the representations of $\sll(2)$.

\subsection{Representations of $U_q(\sll(2))$, $q$ a root of unity}
% Jantzen, 2.11 - 2.13. June 2010 in Clairefontaine book

%% Explain this explicitly
%The representation theory of the associative algebra $U(\sll(2))$ can
%be described in exactly the same way as that of the Lie algebra
%$\sll(2)$. 

%todo: all the content is here I think, but structure it better 

Suppose now that $q$ is a root of unity. Here the situation is considerably
different. The category of representations is no longer semisimple. It is
still possible to classify the irreducible representations, and we will do so. 

First, we will show the following result:

\begin{theorem}
If $q$ is a root of unity, then there are no simple $U_q(\sll(2))$-modules
with dimension $\geq \ell+1$. 
\end{theorem}

\begin{proof}
    %todo
\end{proof}

The irreducible representations of
$U_q(\sll(2))$ can be classified as follows: 

For each $n\geq 0$, there are 2 simple $U_q(\sll(2))$-modules of dimension $n+1$: 
% todo: fix formatting here
$V_+(n)$ with basis $v^+_0, \ldots, v^+_n$ \\
$V_-(n)$ with basis $v^-_0, \ldots, v^-_n$

such that: 
% todo: find a better way to motivate q-notation. (maybe that paper that
% prakash showed you?)
\begin{align*}
    &K v^+_i = q^{n-2i} v^+_i  &
    &K v^-_i = -q^{n-2i} v^-_i \\
    &F v^+_i = \begin{cases} [i+1]_qv_{i+1}& \text{ if $i < n$} \\ 0& \text{ if $i = n$} \end{cases} &
    &F v^-_i = \begin{cases} [i+1]_qv'_{i+1}& \text{ if $i < n$} \\ 0& \text{ if $i = n$} \end{cases} \\
    &E v^+_i = \begin{cases} 
                     [n-i+1]_qv_{i-1}& \text{ if $i > 0$} \\ 
                    0& \text{ if $i = 0$} 
             \end{cases} &
    &E v^-_i = \begin{cases} 
                    -[n-i+1]_q v'_{i-1}& \text{ if $i > 0$} \\ 
                    0& \text{ if $i = 0$} 
             \end{cases}
\end{align*}

% todo: why are there 2 representations here instead of 1? what extra freedom
% allows that to happen?

Suppose $\ell$ is odd, and $q$ is an $\ell^{\text{th}}$ root of unity. 

If $0 \leq n \leq \ell - 1$, the $(n+1)$-dimensional
$U_q(\sll(2))$-modules are the $V_{\pm}(n)$ described above. 

It remains to classify the $\ell$-dimensional $U_q(\sll(2))$-modules.
There are $3$ infinite classes of simple $\ell$-dimensional representations of  $U_q(\sll(2))$:
\begin{enumerate}
        % todo: say somewhere that the results and notation are taken from
        % Jantzen
        \item For any $b,\lambda \in \mathbf{C}, \lambda \neq 0$, define
            $Z_b(\lambda)$ to be the module with basis $v_0, \ldots, v_{\ell -
            1}$ with action of $U$ given by: 
\begin{align*}
    &K v_i = q^{-2i} \lambda v_i \\
    &F v_i = \begin{cases} v_{i+1}& \text{ if $i < n$} \\  b v_0& \text{ if $i = n$} \end{cases}  \\
    &E v_i = \begin{cases} 
        [i]_q \frac{\lambda q^{1-i} - \lambda^{-1} q^{i-1}}{q - q^{-1}} v_{i-1}& \text{ if $i > 0$} \\ 
                    0& \text{ if $i = 0$} 
             \end{cases} 
\end{align*}

If $b \neq 0$ or $\lambda^{2\ell}\neq 1$ then $Z_b(\lambda)$ is simple.
$Z_0(\pm q^k)$ is simple if and only if $k = \ell - 1$. 

\begin{remark}
    $U_q(\sll(2))$ is not semisimple: the modules $Z_0(\pm q_k)$ are not semisimple for $0 \leq k < \ell - 1$.
\end{remark}
\item For any $U_q(\sll(2))$-module $N$, define $^\omega N$ to be
    equal to $N$ as a vector space, and where $u$ acts on $^\omega N$ as
    $\omega(u)$ acts on $N$.

    The $^\omega Z_b(\lambda)$ are another class of modules.
\item Let $a,b,\lambda \in \mathbf{C}$, $a,b\neq 0$
    % todo: say something about them not all being the same.
\begin{align*}
    &K v_i = q^{-2i} \lambda v_i \\
    &F v_i = \begin{cases} 
                v_{i+1}& \text{ if $i < n$} \\  
                b v_0& \text{ if $i = n$} 
             \end{cases}  \\
    &E v_i = \begin{cases} 
                \left(ab + [i]_q \frac{(\lambda q^{1-i} - \lambda^{-1} q^{i-1})}{q - q^{-1}}\right) v_{i-1}& \text{ if $i > 0$} \\ 
                a v_{\ell - 1}& \text{ if $i = 0$} 
             \end{cases} 
\end{align*}
\end{enumerate}


\subsection{Representations of $U_q(\mathfrak{g})$}
    \subsubsection{$U_q(\mathfrak{g})$}

        % taken from Jantzen

        Let $\mathfrak{g}$ be an arbitrary semisimple algebra. Suppose $\Pi$ is
        a basis of the root system, and denote the entries of the Cartan marix
        by 

        \begin{equation}
            a_{\alpha\beta} = 2(\alpha, \beta) / (\alpha, \alpha).
        \end{equation}

        Then the Lie algebra has a presentation with $3|\Pi|$ generators $x_\alpha, y_\alpha, h_\alpha$, $\alpha \in \Pi$ and the relations 

        \begin{align}
            \left[ h_\alpha, h_\beta \right] &= 0       &   \left[ x_\alpha, y_\beta \right] &= \delta_{\alpha\beta} h_\alpha \\
            \left[ h_\alpha, x_\beta \right] &= a_{\alpha\beta} x_\beta  &   \left[ h_\alpha, y_\beta \right] &= -a_{\alpha\beta} y_\beta \\
        \end{align}

        When constructing the algebra $U_q(\sll(2))$, we required that $q^2 \neq 1$. We need a stlightly stronger version of this in general: set

        \begin{equation}
            d_\alpha = \frac{(\alpha, \alpha)}{2}
        \end{equation}

        We require that $q^{2d_{\alpha}} \neq 0$ for all $\alpha$. This means
        that $q^2 \neq 0$, may also require that $q^4, q^6 \neq 0$, depending
        on the length of the longest root in $\mathfrak{g}$.

        Now set 

        \begin{equation}
            q_\alpha = q^{d_\alpha}
        \end{equation}

        and for $n \in \mathbf{Z}$,

        \begin{equation}
            [n]_\alpha = [n]_{q_\alpha} = \frac{q^{nd_\alpha} - q^{-nd_\alpha}}{q^{d_\alpha} - q^{-d_\alpha}}
        \end{equation}

        $[n]_\alpha^!$ and $\dbinom{n}{k}_\alpha$ are defined similarly.

        \begin{defn}
            The quantized enveloping algebra $U_q(\mathfrak{g})$ has generators $E_\alpha, F_\alpha, K_\alpha, K_\alpha^{-1}$ for each $\alpha \in \Pi$, and relations

            \begin{align}
                K_\alpha K_\alpha^{-1} =\ &1  = K_\alpha^{-1}K_\alpha \\
                K_\alpha K_\beta &= K_\beta K_\alpha \\
                K_\alpha E_\beta K_\alpha^{-1} &= q^{(\alpha, \beta)} E_\beta \\
                K_\alpha F_\beta K_\alpha^{-1} &= q^{-(\alpha, \beta)} E_\beta \\
                [E_\alpha, F_\beta] &= \delta_{\alpha\beta} \frac{K_\alpha - K_\alpha^{-1}}{ q_\alpha - q_\alpha^{-1}} 
            \end{align}

            for all $\alpha, \beta \in \Pi$, and for $a \neq \beta$

            \begin{align}
                \sum_{s=0}^{1-a_{\alpha\beta}} (-1)^s \dbinom{1-a_{\alpha\beta}}{s}_\alpha E_\alpha^{1-a_{\alpha\beta} - s} E_\beta E_\alpha^s  &= 0 \\
                \sum_{s=0}^{1-a_{\alpha\beta}} (-1)^s \dbinom{1-a_{\alpha\beta}}{s}_\alpha F_\alpha^{1-a_{\alpha\beta} - s} F_\beta F_\alpha^s  &= 0 
            \end{align}
        \end{defn}
        \subsubsection{A Hopf algebra structure on $U_q(\mathfrak{g})$}

        We can define a Hopf algebra structure $(\Delta, \varepsilon, S)$ such that for all $\alpha \in \Pi$,

        \begin{align}
            \Delta(E_\alpha) &= E_\alpha \otimes 1 + K_\alpha \otimes E_\alpha      & \varepsilon(E_\alpha) &= 0  & S(E_\alpha) &= -K_\alpha^{-1} E_\alpha \\
            \Delta(F_\alpha) &= F_\alpha \otimes K_\alpha^{-1} + 1 \otimes F_\alpha & \varepsilon(F_\alpha) &= 0  & S(F_\alpha) &= -F_\alpha K_\alpha \\
            \Delta(K_\alpha) &= K_\alpha \otimes K_\alpha                           & \varepsilon(K_\alpha) &= 1  & S(K_\alpha) &= K_\alpha^{-1}
        \end{align}
    \subsubsection{Representation theory of $U_q(\mathfrak{g})$}

\section{Modular Tensor Categories from $U_q(\mathfrak{g})$}
% from Bakalov & Kirillov, section 3.3
We will denote the category of representations of $U_q(\mathfrak{g})$ at a
$\ell^\text{th}$ root of unity $q$ by $\mathcal{C}(\mathfrak{g}, \ell)$. 

This is a rigid monoidal category because of the Hopf algebra structure on
$U_q(\mathfrak{g})$ described above. 

As discussed in the previous section, $\mathcal{C}(\sll(2),\ell)$ is not
semisimple, and further has an infinite number of nonisomorphic
$\ell$-dimensional simple objects.  

The same is true for the category $\mathcal{C}(\mathfrak{g}, \ell)$ for a
general semisimple Lie algebra $\mathfrak{g}$.

Our goal is to find an appropriate semisimple part of this category to obtain a
modular tensor category as a quotient.




\subsection{Quantum trace}

The Hopf algebra structure on $U_q(\mathfrak{g})$, coupled with the $R$-matrix,
gives the category of representations $\mathcal{C}(\mathfrak{g}, \ell)$ the
structure of a ribbon category. There is therefore a well-defined trace of an
endomorphism in the category. This is distinct from the usual trace of an
vector space endomorphism. 

We will denote the trace of a morphism $f$ by $\tr_q(f)$.

In the case $U_q(\sll(2))$, the quantum trace is given by 

\begin{equation}
    \tr_q(f) = \tr(Kf)
\end{equation}

where $\tr$ is the usual trace of a vector space endomorphism.

\subsection{Negligible morphisms}
% todo: sort out which things should be theorems and which things should be
% lemmas
To make this subcategory into a modular tensor categories, we will need to
quotient this category by the modules with quantum dimension zero. This is a
fairly natural thing to do.  % todo: why, really?

\begin{defn}
    A morphism $f: T_1 \to T_2$ is \emph{negligible} if $\tr_q(fg) = 0$ for any $g: T_2 \to T_1$
\end{defn}


\begin{defn}
    A module $T$ is \emph{negligible} if all of its endomorphisms are negligible.
\end{defn}
\subsection{Tilting modules} 
% when citing, say something like: originally done by Anderson & P, there's
% something more accessible by Sawin

The first step is to restrict our attention to the subcategory of \emph{tilting
modules}. We will define tilting modules, describe some results classifying the
tilting modules in $U_q(\mathfrak{g})$, and finally list the tilting modules in
the category $\mathcal{C}(\sll(2), \ell)$.

% todo: define Weyl modules
\begin{defn}
    Suppose $M$ is a $U_q(\mathfrak{g})$-module. A \emph{Weyl filtration} $M$ is a sequence of submodules 

    \begin{equation}
        \left\{ 0 \right\} = J_0 \subset \cdots \subset J_n = M
    \end{equation}

    such each $J_k$ is a maximal submodule of $J_{k+1}$ and each quotient $J_{k+1}/J_k$ is a Weyl module. 
\end{defn}

\begin{defn}
    A $U_q(\sll(2))$-module $M$ is called \emph{tilting} if both $M$ and $M^*$ have Weyl filtrations.
\end{defn}


\begin{theorem}
    Every tilting module is a direct sum of indecomposable tilting modules.
    Every indecomposable tilting module is isomorphic to some $T_\lambda$, the
    unique indecomposable tilting module with a maximal vector of weight
    $\lambda$.
\end{theorem}



\begin{lemma}
    A tilting module $T$ is negligible if and only if its quantum dimension is 0.
\end{lemma}

%todo: actually define this region
Define the region $\Lambda^\ell$.

The following two theorems provide a characterization of the tilting modules sufficient for our purposes.

\begin{theorem}
For any $\lambda \notin \Lambda^\ell$, $T^\lambda$ is negligible. 
\end{theorem}

\begin{theorem}
For any $\lambda \in \Lambda^\ell$, $T^\lambda = W^\lambda$.
\end{theorem}



\subsection{Construction of the MTC}

We construct the modular tensor category that we will use to do topological
computing by restricting our attention to the subcategory of tilting modules of
$U_q(\mathfrak{g})$, and quotienting by the negligible morphisms. We can do
this because the hom sets $Hom(V,W)$ are vector spaces:

%todo: change the notation here, I guess
\begin{defn}
    Define the category $\mathcal{C}^\text{int}$ be the category with objects tilting modules and morphisms 

    \begin{equation*}
        \Hom(V,W) = \Hom_T(V,W) / \text{negligible morphisms}
    \end{equation*}

\end{defn}

The category $\mathcal{C}^\text{int}$ has the following properties:
\begin{enumerate}
    \item An object $T$ is negligible if and only if it is isomorphic to 0 in $\mathcal{C}^\text{int}$
    \item  $\mathcal{C}^\text{int}$ is a ribbon category
    \item Any object $T$ in $\mathcal{C}^\text{int}$ is isomorphic to a direct sum of Weyl modules.
    \item $\mathcal{C}^\text{int}$ is a semisimple abelian category. 
    \item $\dim_{\mathcal{C}^\text{int}} T > 0$ for every $T \not\simeq 0$
\end{enumerate}

$\mathcal{C}^\text{int}$ is in fact a modular tensor category. A proof of this
can be found in Bakalov \& Kirillov. %todo: make this a real reference
 
\subsection{Fusion rules}

The fusion rules for $U_q(\sll(2))$ are given by 

%todo: notation is inconsistent here
\begin{equation}
    V_m \otimes V_n \simeq \sum_i N_{mn}^i V_i
\end{equation}

where 

\begin{equation}
    N_{mn}^i = \begin{cases} 1 \text{ for } |m-n| \leq i \leq m+n, i \leq 2k - (m+n), i + m + n \in 2 \mathbf{Z} \\
                             0 \text{ else } 
               \end{cases}
\end{equation}

% Bakalov & Kirillov

\section{From MTC to TQC}
\subsection{Overview of TQC}
\subsection{The Fibonacci anyon}
\subsection{Universality}

\bibliographystyle{plain}
\bibliography{library}

\end{document}
