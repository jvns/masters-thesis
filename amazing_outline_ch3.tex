% Document Type: LaTeX
\documentclass[10pt]{article}
\usepackage{a4}
\usepackage[parfill]{parskip}
\usepackage{amsmath,amssymb,latexsym,stmaryrd}
%\usepackage{epsfig,epstopdf}
\usepackage[all]{xy}
\usepackage[usenames]{color}

\begin{document}
\bibliographystyle{alpha}
\title{Chapters 3, 4 and 5}
\author{Julia Evans}
\date{$9^{th}$ March 2011}
\maketitle
\setcounter{section}{2}
\section{The Representation Theory of $U_q(\mathfrak{sl}(2))$}
\subsection{$U_q(\mathfrak{sl}(2))$ as a deformation of the universal enveloping
  algebra of $\mathfrak{sl}(2)$}

This part spells out the algebraic structure of $U_q(\mathfrak{sl}(2))$.
It shows how it arises as a deformation, i.e. as $q\to 1$ we get
$\mathfrak{sl}(2)$.  
\subsection{A Hopf algebra structure on $U_q(\mathfrak{sl}(2))$}
\subsection{Representations when $q$ is not a root of unity}
This short section essentially conveys the message that the representation
theory here is virtually the same as for $\mathfrak{sl}(2)$.  Proofs will
not be given here.
\subsection{Representation theory when $q$ is a root of unity}
Point out what changes: the centre is different, the raising operators are
no longer nilpotent in all the representations.  The representations are
not completely indexed by the weights any more.

Construct and classify the representations.  With proofs as far as
possible.
\subsection{$R$-matrix}

\subsection{Representation theory for $U_q(\mathfrak{g})$}
\subsubsection{$U_q(\mathfrak{g})$}
\subsubsection{Representation theory of $U_q(\mathfrak{g})$}
Definitions and what the theory looks like without proofs.

\section{Modular Tensor Categories from $U_q(\mathfrak{g})$}
Overview of the strategy.  We will define things for the general case as
far as possible but give detailed calculations and explicit formulas for
the case where $\mathfrak{g} = \mathfrak{sl}(2)$.

\subsection{Tilting Modules} Define: Weyl modules, Weyl filtration, tilting
modules

\subsection{Negligible morphisms} Write explicit formula for the quantum
trace. 

\subsection{Construction of the MTC}
How much detail is up in the air at the moment.  This will certainly not be
done for the general case.

\subsection{Fusion rules}
They come from the $S$-matrix by the Verlinde formula.  Say what the fusion
rules are in $U_q(\mathfrak{sl}(2))$.

\section{From MTC to TQC}
\subsection{Overview of TCQ}
Somehow Julia will manage to do this without mentioning physics!  How are
qubits encoded mathematically.  
\subsection{The Fibonacci anyon}
How does this category come from the representations of
$U_q(\mathfrak{sl}(2))$.  Bonesteel's argument that 2 simple objects are
essentially the same.  This will be also done without any physics.
\subsection{Universality}


\end{document}
