Snakes are elongate, legless, carnivorous reptiles of the suborder Serpentes
that can be distinguished from legless lizards by their lack of eyelids and
external ears. Like all squamates, snakes are ectothermic, amniote vertebrates
covered in overlapping scales. Many species of snakes have skulls with many
more joints than their lizard ancestors, enabling them to swallow prey much
larger than their heads with their highly mobile jaws. To accommodate their
narrow bodies, snakes' paired organs (such as kidneys) appear one in front of
the other instead of side by side, and most have only one functional lung. Some
species retain a pelvic girdle with a pair of vestigial claws on either side of
the cloaca.

Living snakes are found on every continent except Antarctica and on most
islands. Fifteen families are currently recognized, comprising 456 genera and
over 2,900 species.[1][2] They range in size from the tiny, 10 cm-long thread
snake to pythons and anacondas of up to 7.6 metres (25 ft) in length. The
recently discovered fossil Titanoboa was 15 metres (49 ft) long. Snakes are
thought to have evolved from either burrowing or aquatic lizards during the
Cretaceous period (c 150 Ma). The diversity of modern snakes appeared during
the Paleocene period (c 66 to 56 Ma).

Most species are nonvenomous and those that have venom use it primarily to kill
and subdue prey rather than for self-defense. Some possess venom potent enough
to cause painful injury or death to humans. Nonvenomous snakes either swallow
prey alive or kill by constriction.
