\section{Overview of TQC}

\begin{itemize}
\item motivation (more robust implementation)
    \item Kitaev etc (find references)
    \item Setup (anyons, exchanging, fusing)
    \item worldlines as braids, dependence only on topological class of braid
    \item TQC is modelled by MTCs
    \item a couple of words about current experimental status
\end{itemize}
%First we'll discuss the basic idea behind TQC.
%
%One of the fundamental problems with implementations of quantum computer is
%that it is very hard to implement quantum gates accurately. Kitaev in 1997
%propoesd an implementation of QC which exploits topological properties to make
%this easier.
%
%In our basic setup, we have a set of identical anyons which have some joint ``state''.
%
%\begin{center}
%    (picture here?)
%%    $\circ \hspace{1in} \circ\hspace{1in} \circ\hspace{1in} \circ$
%\end{center}
%
%A computation consists of exchanging the anyons in some way. If we draw such an
%exchange in 2+1 dimensions, we obtain a braid. 
%
%\begin{center}
%    (picture here?)
%\end{center}
%
%The key point here is that in this model the change in state of the anyons
%depends \emph{only} on the topological class of the braid, so minor
%pertubations of a particle's path won't affect the result of the computation at
%all. The process is thus inherently resistant to errors. 


%\expand{A summary of the notions behind TQC. Bosonic and fermionic statistics. Introduction of the term ``anyon''. Give some history of the subject. Some of the results that TQC has given us (Jones polynomial universal for QC)}
%
%In \expand{year}, \expand{who?} found that certain quasiparticles in fractional
%quantum Hall states had statistics which were neither fermionic nor bosonic. In
%particular two anyon models which are nonabelian have been proposed: the Ising
%and Fibonacci anyons. There are modelled by the MTCs obtained from the quantum
%groups $U_q(\sll(2))$ at a fourth and fifth root of unity respectively. 
%
%Anyons are essentially described by MTCs. 
%
%\expand{Anyons are modelled by MTCs, and these MTCs are obtained from quantum groups at roots of unity}
%
%\todo{Prakash: Rowell says `only anyonic properties of bosonic systems can be described fully by MTCs'. Can you make some sense of this?}
%
The hope is to give a concrete account of the theory of modular tensor
categories obtained from quantum groups at roots of unity for computer
scientists, from a category theoretic and algebraic point of view rather than a
physical point of view. 

